%%%  Syllabus template modified from http://kjhealy.github.com/latex-custom-kjh
\documentclass[10pt,article,oneside]{memoir}

\usepackage{fontspec} % for xelatex
\usepackage{longtable} % for the course calendar
\usepackage{layouts}[2001/04/29]
\usepackage[notes, 
%	isbn=false,
%	url=false,
% notefield=false,
%	bookseries=false,
	includeall=false, % takes care of the above
	doi=true,
%	citereset=chapter,
%	maxbibnames=2, % only works in the bib and not the footcite.
%	minbibnames=1,
	short,
	backend=biber]{biblatex-chicago} % Adds biblatex.
\usepackage[breaklinks=true]{hyperref}



\def\mytitle{Introduction to the Study of Literature}
\def\instructor{Moacir P. de Sá Pereira}
\def\instructorurl{http:/\slash moacir.com}
\def\instructoremail{blah@blah.com}
\def\courseurl{http:/\slash moacir.com\slash courses-nyu\slash english-101-2017}
\def\coursecode{ENGL-UA 101.001}
\def\courseterm{Spring 2017}
\def\courseroom{ARC LL03}
\def\coursetime{MW 15:30–16:45}
\def\officehours{244 Greene, 506, T 15:00–16:30}
\def\githuburl{http:/\slash github.com\slash muziejus\slash english-101-2017}
\def\transcludebase{sections}
% For the pdf
\hypersetup{pdftitle={\mytitle}, pdfauthor={\instructor}}

\nocite{eat-the-document, gatsby-trade, pynchon-paranoia, lot-49, gatsby-guide, felski-limits, paranoid-reading, latour-critique, python-nltk-cookbook, reading-machines, atp-massumi, reeve-python-nltk, nltk-book}
%\bibliocommand

%	8.5 x 11 layout for memoir-based documents
% Taken from the multimarkdown template.

% This assumes US Letter paper.

%%% need more space for ToC page numbers
\setpnumwidth{2.55em}
\setrmarg{3.55em}

%%% need more space for ToC section numbers
\cftsetindents{part}{0em}{3em}
\cftsetindents{chapter}{0em}{3em}
\cftsetindents{section}{3em}{3em}
\cftsetindents{subsection}{4.5em}{3.9em}
\cftsetindents{subsubsection}{8.4em}{4.8em}
\cftsetindents{paragraph}{10.7em}{5.7em}
\cftsetindents{subparagraph}{12.7em}{6.7em}

%%% need more space for LoF numbers
\cftsetindents{figure}{0em}{3.0em}

%%% and do the same for the LoT
\cftsetindents{table}{0em}{3.0em}

%%% set up the page layout
\settrimmedsize{\stockheight}{\stockwidth}{*}	% Use entire page
\settrims{0pt}{0pt}

\setlrmarginsandblock{1in}{1in}{*}
\setulmarginsandblock{1in}{1.2in}{*}

\setmarginnotes{17pt}{51pt}{\onelineskip}
\setheadfoot{\onelineskip}{2\onelineskip}
\setheaderspaces{*}{2\onelineskip}{*}
\checkandfixthelayout



\begin{document}

% Get some fonts for xelatex.
% EB Garamond is available here: https://www.google.com/fonts/specimen/EB+Garamond
% It has no bold form, so Helvetica stands in, instead
% DejaVu is a whole family of fonts, and I love the mono version: http://dejavu-fonts.org/wiki/Main_Page
 % \setromanfont[Mapping=tex-text,BoldFont={Helvetica Bold}]{EB Garamond} 
 % \setsansfont[Mapping=tex-text]{Helvetica} 
 % \setmonofont[Mapping=tex-text,Scale=0.8]{DejaVu Sans Mono}

\title{\LARGE {\normalsize \textsc{\coursecode}\\} \HUGE \mytitle \\ \Large\url{\courseurl}}     
\author{\Large\instructor\\ \small\texttt{\noindent\instructoremail}}
\date{\courseterm. \courseroom \\ \coursetime \\ \small Office hours: \officehours}

\maketitle

% Main Content


\{\{navbar.*\}\}

\chapter{Course description}
\label{coursedescription}

Over the past decade, literary study has become increasingly, reflexively
interested in investigating the methods that it generates. At the same time,
the digital humanities have emerged to claim a part of growing English
departments. This course serves as an introduction to both of these currents in
contemporary literary study. We will consider both familiar forms of reading as
well as new, different forms that have broadened the way we interpret texts,
and we will then look to how the digital humanities, as an especially
method-oriented subfield, brings these questions of interpretation into even
sharper contrast. Students will, then, see how debates about interpretation are
lived and experienced within the digital humanities, both in theory and
practice. Finally, students will learn to use digital methodologies in their
interpretation, with possible projects in textual analysis, topic modeling, and
geospatial analysis.

\chapter{Goals of the course}
\label{goalsofthecourse}

\begin{itemize}
\item to introduce you to historical trends in literary criticism from the 20th and 21st centuries;

\item to develop skills in

\begin{itemize}
\item reading analytical and literary texts;

\item writing analyses that are cogent and syncretic, making use of the various methods on hand;

\item creating (as well as using and distributing) geospatial datasets,
 GISes, and cartographic visualizations of the former;

\item creating (as well as using and distributing) corpora in Python;

\item using \texttt{NLTK} to investigate the properties of a text;

\end{itemize}

\item to develop, refine, and present scholarship that exists, spatially and
temporally, beyond the boundaries of the course.

\end{itemize}

\chapter{Books}
\label{books}

\begin{itemize}
\item F. Scott Fitzgerald, \emph{The Great Gatsby}, 1925 (Scribner) \href{http://www.strandbooks.com/strand%2D80/great%2Dgatsby/_/searchString/gatsby}{Strand}\footnote{\href{http://www.strandbooks.com/strand\%2D80/great\%2Dgatsby/\_/searchString/gatsby}{http:/\slash www.strandbooks.com\slash strand\%2D80\slash great\%2Dgatsby\slash \_\slash searchString\slash gatsby}}

\item Thomas Pynchon, \emph{The Crying of Lot 49}, 1965 (Harper Perennial) \href{http://www.strandbooks.com/fiction/crying%2Dof%2Dlot%2D49%2D1/_/searchString/crying%20of%20lot%2049}{Strand}\footnote{\href{http://www.strandbooks.com/fiction/crying\%2Dof\%2Dlot\%2D49\%2D1/\_/searchString/crying\%20of\%20lot\%2049}{http:/\slash www.strandbooks.com\slash fiction\slash crying\%2Dof\%2Dlot\%2D49\%2D1\slash \_\slash searchString\slash crying\%20of\%20lot\%2049}}

\end{itemize}

\chapter{Course requirements \& policies}
\label{courserequirementspolicies}

\section{Assessment}
\label{assessment}

\subsection{Participation}
\label{participation}

The success of any course is directly related to the levels of engagement
brought both by the instructor and the students. As such, class participation
is vitally important. Similarly, though attendance is logically required for
class participation, it is not sufficient. This class requires active
participation both inside the classroom and outside. No “passive consumers,” as
a professor of mine put it. 

You can miss up to three meetings without penalty, and you can use these
opportunities tactically, to provide space and time to either fulfill other
obligations or recuperate from the previous night. I don’t care why you didn’t
come. I start to care with the fourth absence, and I start to require
documentation. Repeated unexcused absence quickly gobbles up the class
participation component of the grade and begins to threaten your ability to
even \emph{pass} the course.

Because this course is discussion-oriented, active participation means, most
importantly, participating in the discussions in class. But useful and engaged
participation in discussions also depends on good preparation, which includes
doing the reading for the course. I encourage (but will not collect) you to
think of one or two points of entry into a discussion of a text for each
meeting. This could be a point of confusion (don’t be shy!), a point of
comparison\slash contrast between passages to another work, or a useful parallel to
something outside the coursework. Come to class with questions, in other words,
and writing them out as mini-prompts may be especially helpful.

\subsection{Critical presentation}
\label{criticalpresentation}

While reading the critical history of \emph{The Great Gatsby}, each of you
will give a short (8–10 minutes) presentation that introduces either one of the
essays cited in the reading for that day or an essay you have found via JSTOR
contemporary with that day’s reading. You will get to sign up for a time period
on the first day of class. The presentation will also have a short (1 page)
written component that you will turn in. The goal of the presentation is to
introduce new knowledge to the class that it has not already had, thereby
facilitating that day’s discussion. This presentation cannot make use of the
computer in the classroom, and you should email the subject of your
presentation to me 24 hours in advance.

\subsection{Digital presentation}
\label{digitalpresentation}

During the digital section of the course, each of you will give a short (8–10
minutes) presentation on a digital tool or project you have found online,
possibly even making some quick use of the Gatsby dataset. The tool or
project should be appropriate to that day’s work, meaning statistical analysis
tools and projects fit for the Python days, while spatial analysis
tools and projects fit better for the Carto days. You can sign up
on the first day for these presentations. Your presentation will also have a
short (1 page) written component that you will turn in. The goal of the
presentation is to introduce a new form of digital reading to the class that
may spark new ideas for final projects. This presentation can make use of the
computer in the classroom, and you should email the subject of your
presentation to me 24 hours in advance.

\section{Policies}
\label{policies}

\subsection{Assignments}
\label{assignments}

The assignment instructions, though detailed in the syllabus, may be enhanced
or supplemented during the course. If you have any questions about an
assignment, you should ask for clarification early. The assignments are due on
the dates noted in the syllabus. 

All of the writing can be submitted electronically.

Late assignments jeopardize both your and my rhythms in the class, so they will
be penalized. I will give you feedback and will happily discuss any work with
you, but grades should be considered final.

Additionally, grading is variable based on what you feel your strengths are.
Each assignment will be worth at least 16\% of your final grade, but the upper
limit of the grade is set by you. You should email me how you slice up the pie
by the end of October.

\subsection{Attendance}
\label{attendance}

As indicated above, attendance is required. Three absences will be excused
without supplemental documentation, and I encourage you to use these
tactically. Catching up is your responsibility.

Subsequent absence requires formal documentation. Otherwise it begins to harm
your final grade. Though class participation is only part of the final grade,
extreme absenteeism (more than six meetings missed) may put your ability to
pass the course at risk.

Please show up on time to class, as well.

\subsection{Digital Learning}
\label{digitallearning}

I will be exposing you to a lot of new tools and concepts. Our class will have
digitally-focused classes in our classroom, where we will learn new skills.
These skills are difficult, and I will try to help as much as possible.

\subsection{Electronics}
\label{electronics}

Despite the presence of the digital, especially as the class gets deeper into
the semester, our time in class is meant as a sanctuary from the distractions
of the rest of the world. Furthermore, the class relies on discussion and
engagement, and the front of a laptop screen is a brilliant shield behind which
a student can hide, even unintentionally. During our meetings, then, there can
be no use of electronic devices. Please also set whatever devices you have but
aren’t using to silent mode.

\subsection{Communication}
\label{communication}

Communication is vitally important to the pedagogical process, and this course
depends on clear communication in both directions. If you have questions,
comments, or concerns, the best course of action is to come visit me during my
office hours as noted at the top of this page. If your questions, etc., cannot
wait until then, then clearly you can also email me. I should respond within
48 hours.

This is a new course, meaning that there will be even more unfinished edges
ready to scratch someone than in a typical course. We have a collective goal of
learning, however, so if the unfinished edges get to be overwhelming, I’ll
adjust the parameters of the course appropriately. I’m not out to catch you,
nor is this course a process of grotesque punishment. Please don’t treat it as
such.

Once more, with feeling: \emph{communication is vitally important to the pedagogical
process}. If you have concerns or worries, please let me know about them sooner
rather than later.

\subsection{Disabilities}
\label{disabilities}

If you have a disability, you should register with the Moses Center for
Students with Disabilities (mosescsd@nyu.edu; 726 Broadway, 2nd Floor,
212.998.4980), which can arrange for things like extra time for assignments.
Please inform me \emph{at the beginning of the semester} if you need any special
accommodations regarding the assignments.

\subsection{Academic integrity}
\label{academicintegrity}

Please look at NYU’s \href{http://cas.nyu.edu/page/academicintegrity}{full statement on academic
integrity}\footnote{\href{http://cas.nyu.edu/page/academicintegrity}{http:/\slash cas.nyu.edu\slash page\slash academicintegrity}}. Any instance of
academic dishonesty will result in an F and will be reported to the relevant
dean for disciplinary action. Remember that plagiarism is a matter of fact, not
intention. Know what it is, and don’t do it.

\subsection{Syllabus}
\label{syllabus}

The handsome, printer-friendly \texttt{pdf} version of the syllabus is available \href{https://github.com/muziejus/does-it-work/blob/master/syllabus.pdf}{here}\footnote{\href{https://github.com/muziejus/does-it-work/blob/master/syllabus.pdf}{https:/\slash github.com\slash muziejus\slash does-it-work\slash blob\slash master\slash syllabus.pdf}}.

 \newpage 

\chapter{Schedule}
\label{schedule}

Readings that are not the four books listed above will be available on reserve
or by other means. See the list of references at the end for details.

\section{1. That Old \emph{Gatsby}, That Critique \emph{Gatsby}}
\label{1.thatoldgatsbythatcritiquegatsby}

In the first section of this course, we’ll be returning to a familiar,
canonical work of American 20th century literature, \emph{The Great Gatsby}. Next,
we will follow our own reading of the novel with a look at the novel’s critical
history. 

\begin{itemize}
\item Monday, 5 Sep: No Class.

\item Wednesday, 7 Sep: Introductions and a snippet from Massumi’s “Translator’s Foreword: Pleasures of Philosophy.”

\item Monday, 12 Sep: Fitzgerald, The Great Gatsby, 

\item Wednesday, 14 Sep: Fitzgerald, The Great Gatsby,

\item Monday, 19 Sep: Fitzgerald, The Great Gatsby,

\item Wednesday, 21 Sep: (Critical presentations begin), Tredell, intro \& ch. 1.

\item Monday, 26 Sep: Tredell, chs. 2 \& 3.

\item Wednesday, 28 Sep: Tredell, chs. 4 \& 5.

\end{itemize}

\section{2. A Theoretical Break}
\label{2.atheoreticalbreak}

Next, we take a short break to learn about the stresses offered by these more
critique-driven forms of reading.

\begin{itemize}
\item Monday, 3 Oct: Bersani, “Pynchon, Paranoia, and Literature” and Latour,
“Why Has Critique Run out of Steam? From Matters of Fact to Matters of
Concern.”

\item Wednesday, 5 Oct: Sedgwick, “Paranoid Reading and Reparative Reading, or
You’re So Paranoid, You Probably Think This Essay Is about You” and
selections from Felski, The Limits of Critique.

\end{itemize}

\section{3. The New \emph{Gatsby}, The Digital \emph{Gatsby}}
\label{3.thenewgatsbythedigitalgatsby}

These five weeks serve as an opportunity to learn new methods of literary
criticism, now based in digital tools. We will learn how to use \texttt{Voyant} to
quickly see patterns in the text of \emph{The Great Gatsby}, \texttt{Python} to analyze the
text within a high-level statistical programming environment, and, finally,
\texttt{NYWalker} and \texttt{Carto} to learn how to make maps to analyse the geographical
space of \emph{The Great Gatsby}.

\begin{itemize}
\item Monday, 10 Oct: \emph{No Class.}*

\item Monday, 17 Oct: Introduction to Python

\item Wednesday, 19 Oct: (Digital presentations begin), Python and The Great Gatsby

\item Monday, 24 Oct: Python and The Great Gatsby

\item Wednesday, 26 Oct: Python and The Great Gatsby

\item Monday, 31 Oct: NYWalker and The Great Gatsby

\item Wednesday, 2 Nov: Carto and The Great Gatsby

\item Monday, 7 Nov: Carto and The Great Gatsby

\item Wednesday, 9 Nov: Election Day debriefing

\item Monday, 14 Nov: Carto and The Great Gatsby

\end{itemize}

\section{4. The New Novels, The New Systems}
\label{4.thenewnovelsthenewsystems}

The semester closes with reading two new(er) novels that invite a systematic,
totalized reading. We close with student presentations on their final projects.

\begin{itemize}
\item Wednesday, 16 Nov: Novel presentations begin, Pynchon, chs. 1–3.

\item Monday, 21 Nov: Pynchon, chs. 4 \textbackslash{}\& 5.

\item Wednesday, 23 Nov: No Class.

\item Monday, 28 Nov: Pynchon, ch. 6.

\item Wednesday, 30 Nov: Spiotta, pts. 1 \textbackslash{}\& 2.

\item Monday, 5 Dec: Spiotta, pts. 3 \textbackslash{}\& 4.

\item Wednesday, 7 Dec: Spiotta, pts. 5–7.

\item Monday, 12 Dec: Spiotta, pts. 8 \textbackslash{}\& 9.

\item Wednesday, 14 Dec: Final presentations

\end{itemize}

 \newpage 

\chapter{Calendar}
\label{calendar}

\begin{table}[htbp]
\begin{minipage}{\linewidth}
\setlength{\tymax}{0.5\linewidth}
\centering
\small
\begin{tabulary}{\textwidth}{@{}LLL@{}} \toprule
Week&Monday&Wednesday\\
\midrule
1. 5.9, 7.9&\textbf{No class}&Introductions, Massumi\\
2. 12.9, 14.9&Fitzgerald&Fitzgerald\\
3. 15.3, 17.3&\multicolumn{2}{l}{\textbf{Spring Break}}\\

\midrule
4. 20.3, 22.3&Unit 2&\texttt{electric boogaloo}\\

\bottomrule

\end{tabulary}
\end{minipage}
\end{table}


\end{document}



\end{document}
